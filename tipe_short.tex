\documentclass{beamer}


\usepackage[utf8]{inputenc}
\usepackage{amsmath}
\usepackage{amssymb}
\usepackage{graphicx}
\usepackage{wrapfig}
\usepackage{lipsum}

\usetheme{Warsaw}
\author{Axel Pereyrol}
\date{18/06/23}
\parindent=0pt
\begin{document}

\title{Présentation TIPE}

%--------------------------------------------------------------------------

\section{Problématique}
\begin{frame}
\frametitle{Code correcteurs et Takuzu}


\begin{itemize}
\item \structure{Comment assurer un transfert de données sur un canal peu fiable tout en minimisant la quantité de données envoyé ?}
\end{itemize}

Lorsqu'on désire envoyer une information à une machine distante, il arrive que certaines données soient perdues/effacées.

Pour remédier à cela, des codes correcteurs sont mis en place sur la machine distante, cette méthode est basée sur la redondance (une donnée envoyée plusieurs fois) afin de pouvoir détecter les erreurs et/ou les corriger.
\newline



\begin{minipage}[0.2\textheight]{\textwidth}
\begin{columns}[T]
\begin{column}{0.8\textwidth}
\begin{itemize}
\item \structure{But du TIPE} : poser une contrainte sur la forme d'envoie pour récupérer les données initiales de manière logique : 
\end{itemize}
Encodage de l'information sous forme de séquences de \structure{Takuzu} de tailles variables.
\end{column}
\begin{column}{0.2\textwidth}
\centering
\includegraphics[width=2.5cm]{/home/triplaqs/Documents/MP2I/TIPE/Img/takuzuex.png}
\label{figure}{Un \structure{Takuzu}.}
\end{column}
\end{columns}
\end{minipage}
\newline
\newline
\newline

\end{frame}


%--------------------------------------------------------------------------

\section{Démarche}
\begin{frame} 
\frametitle{Démarche}

Les \structure{3 grandes contraintes} de mon sujet sont les suivantes :

\begin{itemize}
\item Encoder les données sous forme de Takuzu, malgré les règles imposées, tout en conservant leurs informations.
\item Avoir l'unicité de la solution d'un Takuzu et sa CNS sur les cases dévoilées initialement.
\item Savoir où se situent les pertes au sein d'une séquence de donnée.
\end{itemize}

Pour pouvoir continuer \structure{malgré ces contraintes}, voici les démarches imaginées :

\begin{itemize}
\item Commencer par une transformation manuelle sur des petites séquences, avant de trouver un moyen de l'automatiser.
\item Trouver exhaustivement la CNS du nombre de cases initialement dévoilée pour obtenir l'unicité de la solution.
\item Admettre la structure des données, et les pertes sous forment d'inconnu dans un premier temps.
\end{itemize}
\end{frame}

%--------------------------------------------------------------------------

\section{Commencement}
\begin{frame}
\frametitle{Le lancement}
\begin{itemize}
\item \structure{Encodage} :
\end{itemize}
Code Python qui transforme un image en matrice de triplets (R,G,B), puis en cette même matrice de triplets mais en binaire.
Puis conversion en grilles de Takuzu manuellement (En cours d'encodage).
\newline


\begin{minipage}[0.2\textheight]{\textwidth}
\begin{columns}[T]

\begin{column}{0.63\textwidth}

\begin{itemize}
\item \structure{Résolution} :
\end{itemize}

Code C, code qui vérifie les 3 conditions du Takuzu (fonctionnelles).
Code qui vérifie si des pertes ont eu lieu (pertes implémentées à -1).
Code qui résoud la première condition (en cours de test (C1 OK)).
\end{column}

\begin{column}{0.08\textwidth}
\centering
\includegraphics[width=1.5cm]{/home/triplaqs/Documents/MP2I/TIPE/Img/test1.png}
\end{column}

\begin{column}{0.08\textwidth}
\centering
\includegraphics[width=1.5cm]{/home/triplaqs/Documents/MP2I/TIPE/Img/test2.png}
%\label{figure}{Un \structure{Takuzu}.}
\end{column}

\begin{column}{0.08\textwidth}
\centering
\includegraphics[width=1.5cm]{/home/triplaqs/Documents/MP2I/TIPE/Img/test3bis3edit.png}
\end{column}

\end{columns}
\end{minipage}


\begin{itemize}
\item \structure{Preuve d'unicité} (sur une grille de taille raisonnable) :
\end{itemize}
Prévue ces vacances grace à un backtracking utilisant les fonctions implémentées en C.

\end{frame}

%--------------------------------------------------------------------------

\end{document}

%--------------------------------------------------------------------------

%$\bullet$ \structure{Comment assurer un transfert de données sur un canal peu fiable tout en minimisant la quantité de données envoyé ?}