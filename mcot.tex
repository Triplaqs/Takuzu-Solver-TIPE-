\documentclass{beamer}


\usepackage[utf8]{inputenc}
\usepackage{amsmath}
\usepackage{amssymb}
\usepackage{graphicx}
\usepackage{wrapfig}
\usepackage{lipsum}

\usetheme{Warsaw}
\author{Axel Pereyrol}
\date{02/11/23}
\parindent=0pt
\begin{document}

\title{Code correcteurs via Takuzu}

%--------------------------------------------------------------------------

\subsection*{Titre}
\begin{frame} % présentation sujet
\maketitle 
\end{frame}

%--------------------------------------------------------------------------

\subsection*{Sommaire}
\begin{frame}
\frametitle{Sommaire}
\tableofcontents
\end{frame}


%--------------------------------------------------------------------------

%--------------------------------------------------------------------------

\section{Ancrage}
\subsection*{Ancrage}
\begin{frame}
\frametitle{ANCRAGE}

ANCRAGE AU SUJET

\end{frame}


%--------------------------------------------------------------------------

\section{Motivation du choix}
\subsection*{Motivation du choix}
\begin{frame} 
\frametitle{Motivation du choix}

POURQUOI CE SUJET ??

\end{frame}

%--------------------------------------------------------------------------

\section{MCOT}
\subsection{Positionnements thématiques et mots-clés}
\begin{frame}
\frametitle{Positionnements thématiques et mots-clés}
INFO MATHÉMATIQUES PHYSIQUE \\
5 MOTS CLEFS (FR & ANG) :\\ 
-TAKUZU\\
-CODE CORRECTEURS\\
-PERTES\\
-CONTRAINTES\\
-TRANSFERT 
\end{frame}

%--------------------------------------------------------------------------


\subsection{Bibliographie commentée}
\begin{frame}
\frametitle{Bibliographie commentée}



\end{frame}

%--------------------------------------------------------------------------


\subsection{Problématique retenue}
\begin{frame}
\frametitle{Problématique retenue}



\end{frame}

%--------------------------------------------------------------------------


\subsection{Objectifs du TIPE}
\begin{frame}
\frametitle{Objectifs du TIPE}



\end{frame}

%--------------------------------------------------------------------------


\subsection{Liste des références bibliographiques}
\begin{frame}
\frametitle{Liste des références bibliographiques}



\end{frame}


%--------------------------------------------------------------------------



\end{document}



%--------------------------------------------------------------------------


%\begin{itemize}
%\item \structure{blue gras}
%\end{itemize}

%\centering
%\includegraphics[width=2.5cm]{/home/triplaqs/Documents/MP2I/TIPE/Img/takuzuex.png}
%\label{figure}{Un \structure{Takuzu}.}

%\begin{minipage}[0.2\textheight]{\textwidth}
%\begin{columns}[T]
%\begin{column}{0.8\textwidth}
%\end{column}
%\begin{column}{0.2\textwidth}
%\end{column}
%\end{columns}
%\end{minipage}



%--------------------------------------------------------------------------

%$\bullet$ \structure{Comment assurer un transfert de données sur un canal peu fiable tout en minimisant la quantité de données envoyé ?}
