\documentclass{beamer}


\usepackage[utf8]{inputenc}
\usepackage{amsmath}
\usepackage{amssymb}
\usepackage{graphicx}
\usepackage{wrapfig}
\usepackage{lipsum}

\usetheme{Warsaw}
\author{Axel Pereyrol}
\date{02/11/23}
\parindent=0pt
\begin{document}

\title{Code correcteurs via Takuzu}

%--------------------------------------------------------------------------

\subsection*{Titre}
\begin{frame} % présentation sujet
\maketitle 
\end{frame}

%--------------------------------------------------------------------------

\subsection*{Sommaire}
\begin{frame}
\frametitle{Sommaire}
\tableofcontents
\end{frame}


%--------------------------------------------------------------------------

%--------------------------------------------------------------------------

\section{Ancrage}
\subsection*{Ancrage}
\begin{frame}
\frametitle{ANCRAGE}

ANCRAGE AU SUJET

\end{frame}


%--------------------------------------------------------------------------

\section{Motivation du choix}
\subsection*{Motivation du choix}
\begin{frame} 
\frametitle{Motivation du choix}

%POURQUOI CE SUJET ??
-Je me suis toujours intéressé aux CCDD, Reed Solomon, et je cherchais dans ce domaine.\\
-quand j'ai réalisé que l'utilisation de Takuzus et de contraintes sur l'encodage pourraient faire office de CCDD.\\
- Combiner un jeu auquel j'ai énormément joué et développé des stratégies de résolution rapides et un sujet qui m'intéresse depuis plusieurs années en un sujet de TIPE semble être ma meilleure opportunité.

\end{frame}

%--------------------------------------------------------------------------

\section{MCOT}
\subsection{Positionnements thématiques et mots-clés}
\begin{frame}
\frametitle{Positionnements thématiques et mots-clés}
INFO MATHÉMATIQUES PHYSIQUE \\
\vspace{\baselineskip}
5 MOTS CLEFS (FR \& ANG) :\\ 
\vspace{\baselineskip}
-TAKUZU\\
-CODE CORRECTEURS\\
-PERTES\\
-CONTRAINTES\\
-TRANSFERT 
\end{frame}

%--------------------------------------------------------------------------


\subsection{Bibliographie commentée}
\begin{frame}
\frametitle{Bibliographie commentée}
%CONTEXTE SCIENTIFIQUE

%PREMIÈRE RECHERCHE
\begin{itemize}
\item Complexité temporelle de la possible résolution.\\
\end{itemize}
-Étude des moyens actuels de communication et de transfert dans les satellites (par exemple).

%DEUXIÈME RECHERCHE
\begin{itemize}
\item Unicité de la résolution et CNS de cette unicité (si elle existe).\\
\end{itemize}
- Existence pour le sudoku (CNS : 17 valeurs dévoilées)

%TROISIÈME RECHERCHE
\begin{itemize}
\item Fonctionnement des CCDD.\\
\end{itemize}
- Application dans ce domaine.

%PROBLÉMATIQUES :
\begin{itemize}
\item Problématiques :
\end{itemize}
- Temps de résolution légitime ?\\
- Prouver l'unicité.\\
- Localiser les pertes et encodage.\\




\end{frame}

%--------------------------------------------------------------------------


\subsection{Problématique retenue}
\begin{frame}
\frametitle{Problématique retenue}


\center \huge \structure{PEUT-ON UTILISER LE TAKUZU POUR RÉDUIRE LA REDONDANCE DANS LA CORRECTION LORS D'UN TRANSFERT DE DONNÉE ?}



\end{frame}

%--------------------------------------------------------------------------


\subsection{Objectifs du TIPE}
\begin{frame}
\frametitle{Objectifs du TIPE}

- Mettre en place la résolution d'un Takuzu\\
- Prouver l'unicité à l'aide d'un BackTracking\\
- Mettre en place l'encodage et le désencodage d'une donnée via cette méthode (petite image)\\
- voir les limites\\


\end{frame}

%--------------------------------------------------------------------------


\subsection{Liste des références bibliographiques}
\begin{frame}
\frametitle{Liste des références bibliographiques}

-Sudoku preuve.\\
-Données Satelittes envoie informations.\\



\end{frame}


%--------------------------------------------------------------------------



\end{document}



%--------------------------------------------------------------------------


%\begin{itemize}
%\item \structure{blue gras}
%\end{itemize}

%\centering
%\includegraphics[width=2.5cm]{/home/triplaqs/Documents/MP2I/TIPE/Img/takuzuex.png}
%\label{figure}{Un \structure{Takuzu}.}

%\begin{minipage}[0.2\textheight]{\textwidth}
%\begin{columns}[T]
%\begin{column}{0.8\textwidth}
%\end{column}
%\begin{column}{0.2\textwidth}
%\end{column}
%\end{columns}
%\end{minipage}

%\vspace{\baselineskip}


%--------------------------------------------------------------------------

%$\bullet$ \structure{Comment assurer un transfert de données sur un canal peu fiable tout en minimisant la quantité de données envoyé ?}
